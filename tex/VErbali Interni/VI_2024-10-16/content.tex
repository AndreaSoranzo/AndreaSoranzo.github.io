% Insert content here
\section{Informazioni generali}
\subsection{Dettagli sull'incontro}
\begin{itemize}
    \item \textbf{Luogo}: Incontro in remoto su Google Meet
    \item \textbf{Data}: 16-10-2024
    \item \textbf{Ora di inizio}: 15:30
    \item \textbf{Ora di fine}: 16:30
    \item \textbf{Partecipanti}:
    \begin{itemize}
        \item Bergamin Elia
        \item Chilese Elena
        \item Diviesti Filippo
        \item Djossa Edgar
        \item Pincin Matteo 
        \item Soranzo Andrea 
    \end{itemize}
\end{itemize}

\section{Resoconto}
\subsection{Introduzione}
    Nella prima riunione di gruppo ci siamo concentrati sulla definizione delle informazioni, degli strumenti e dei metodi di lavoro da adottare per l’intera durata del progetto, al fine di garantirne un’organizzazione ottimale. Abbiamo inoltre iniziato a discutere i dettagli relativi ai vari capitolati.
\subsection{Nome Gruppo}
    Dopo una breve ricerca online per raccogliere alcuni spunti, ci siamo soffermati sul nome\textit{Bit Busters}, che ha ottenuto fin da subito l’approvazione unanime di tutti i membri. Successivamente, è stato proposto di anteporre al nome il numero \textit{Six} per rappresentare il numero dei componenti del gruppo, una scelta accolta da tutti con apprezzamento.
\subsection{Logo Six Bit Busters}
    Il logo è stato realizzato da Soranzo Andrea con l'aiuto di \textit{Adobe Illustrator online} e subito approvato da tutto il gruppo. Successivamente si è deciso di creare anche una seconda versione \textit{inline} da utilizzare nelle varie pagine della documentazione. 
\subsection{Email}
    Per comunicare con le aziende ed avere altresì un sistema unificato, abbiamo deciso di creare la mail di gruppo \textit{6bitbusters@gmail.com}. Essa è sfruttata anche per i servizi Google annessi quali \textit{Google Docs} (per note informali) e \textit{Google Calendar} (per pianificare attività e scadenze).
\subsection{Sistema di versionamento}
    Come sistema di versionamento abbiamo scelto fin da subito quello attualmente più diffuso, ovvero \textit{Git}, insieme a \textit{GitHub}, dove, al termine del progetto, verrà creata una repository pubblica contenente tutta la documentazione e il codice del software.
\subsection{Redazione documenti}
    Per la stesura dei documenti, si è scelto di utilizzare \textit{LaTeX} con template appositamente creati a fine riunione da Chilese Elena, al fine di conferire uno stile omogeneo e più professionale alla documentazione.
\subsection{Scelta dei Capitolati}
    Durante questa riunione abbiamo iniziato ad analizzare e discutere i vari capitolati. \\È emersa una netta preferenza comune per il \textbf{capitolato 5} (\textit{\textbf{3Dataviz}}) presentato dalla \textit{\textbf{Sanmarco Informatica}}. \\Abbiamo inoltre riscontrato un interesse comune per i capitolati \textbf{2} (\textit{\textbf{VIMAR GENIALE}}) e \textbf{3} (\textit{\textbf{Automatizzare le routine digitali tramite l’intelligenza generativa}}).
\section{Prossimi obiettivi}
    \begin{itemize}
        \item Effettuare un’analisi più approfondita dei capitolati, evidenziando i relativi pro e contro.
        \item Organizzare un incontro con i proponenti
    \end{itemize}
