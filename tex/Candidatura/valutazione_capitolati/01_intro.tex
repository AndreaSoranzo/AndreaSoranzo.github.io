\section{Studio di fattibilità}

\setcounter{subsection}{0}
\subsection{Introduzione}
Sotto elencate vi saranno le motivazioni che hanno portato il gruppo Six Bit Busters a scegliere il capitolato CX.

\subsection{C1: Artificial QI }
Sviluppo di un sistema per valutare la capacità dei modelli di Intelligenza Artificiale, in particolare Large Language Models (LLM), di rispondere a domande complesse, integrando archiviazione di domande, gestione delle API, valutazione delle risposte, e visualizzazione dei risultati in un'applicazione completa. \\\\
\textbf{Pro}
\begin{itemize}
    \item Tema del progetto attuale, riguarda argomenti che evolveranno in qualcosa di rivoluzionario.
    \item L'azienda accetta proposte innovative.
\end{itemize}
\textbf{Contro}
\begin{itemize}
    \item Non ci sono indicazioni su che tecnologie e linguaggi utilizzare.
    \item L'azienda non specifica il supporto che darà ai gruppi.
\end{itemize}

\subsection{C2: Vimar GENIALE}
Sviluppo di un'applicazione che supporti gli installatori nella ricerca di informazioni tecniche sui prodotti domotici Vimar, utilizzando un'interfaccia di linguaggio naturale e intelligenza artificiale. L'applicativo sarà cloud-ready e offrirà funzionalità di estrazione dati, risposta alle domande e visualizzazione di schemi elettrici e immagini dei dispositivi.\\\\
\textbf{Pro}
\begin{itemize}
    \item Presentazione ben fatta e dettagliata, che rende più facile l'analisi dei requisiti, e rende il progetto più interessante a primo impatto.
    \item Tecnologie e temi diversi che comportano un aumento di competenze utili nel mondo del lavoro.
    \item L'azienda ha specificato l'aiuto che darà ai gruppi.
\end{itemize}
\textbf{Contro}
\begin{itemize}
    \item Sono richiesti almeno 2 incontri in presenza con il gruppo per la consegna dei materiali e il
    collaudo della soluzione. Ciò comporta un aumento dei costi in termini di tempo rispetto a incontri da remoto.
\end{itemize}

\subsection{C3: Automatizzare le routine digitali tramite l’intelligenza generativa}
Un servizio ad agenti dove gli utenti possono disegnare
localmente un workflow sfruttando le API dei software
locali e l’intelligenza artificiale in cloud per automatizzare
attività quotidiane che l’utente svolge manualmente.  Integrato con un sistema di automazione workflow tramite API e intelligenza artificiale su AWS. \\\\
\textbf{Pro}
\begin{itemize}
    \item Alla presentazione del capitolato l'azienda ci è sembrata molto disponibile a seguirci durante lo sviluppo del progetto.
    \item Progetto utile nella vita di tutti i giorni che grazie alle automazioni può essere di beneficio ad una vasta gamma di persone.
    \item Il team fornirà attività di formazione sulle principali tecnologie necessarie allo sviluppo e materiali accessori.
\end{itemize}
\textbf{Contro}
\begin{itemize}
    \item L'appalto è spiegato in maniera poco dettagliata e quindi rende difficoltosa l'analisi dei requisiti.
\end{itemize}


\subsection{C4: NearYou - Smart custom advertising platform}
Il progetto "NearYou" propone una piattaforma di pubblicità personalizzata basata su intelligenza artificiale, in grado di generare e visualizzare annunci dinamici in base alla posizione geografica e al profilo dell'utente, migliorando l'efficacia degli annunci per gli inserzionisti e l'esperienza degli utenti.\\\\
Il progetto presentato da questa azienda non ha riscontrato particolare interesse da parte del gruppo. 

\subsection{C5: 3Dataviz}
Il progetto mira a sviluppare un'interfaccia web per la visualizzazione interattiva di dati in un istogramma 3D, utilizzando coordinate (x, y, z) per rappresentare valori e colori. Saranno supportate diverse modalità di reperimento dati e l'architettura sarà scalabile per aggiungere nuove funzionalità, garantendo al contempo robusti test di unità e integrazione.\\\\
\textbf{Pro}
\begin{itemize}
    \item Presentazione che spiega in poche parole e in maniera diretta cosa tratta la data visualisation.
    \item Progetto che lascia spazio ad iniziative per quanto riguarda il design, in senso estetico, dell'applicativo.
    \item MVP chiaro, ben spiegato ciò che la web app deve essere in grado di sviluppare.
    \item Progetto ritenuto molto stimolante da tutti i componenti del gruppo.
\end{itemize}



\subsection{C6: Sistema di gestione di un magazzino distribuito}
Il progetto mira a sviluppare un sistema di gestione distribuita per ottimizzare l'inventario in una rete logistica di magazzini geograficamente distribuiti. Questo sistema dovrà garantire la sincronizzazione in tempo reale dei dati, il riassortimento predittivo tramite algoritmi di machine learning, e la gestione autonoma delle operazioni di magazzino.\\\\
\textbf{Pro}
\begin{itemize}
    \item Sistema che promuove un ambiente di gestione innovativo e collaborativo per una gestione più fluida e automatizzata delle operazioni logistiche.
    \item L'azienda specifica l'aiuto proposto nella fase di analisi.
\end{itemize}
\textbf{Contro}
\begin{itemize}
    \item Il contesto in cui il progetto si pone non ci ha particolarmente incuriosito e la complessità del progetto ci ha portato ad intraprendere scelte diverse.
    \item L'azienda non specifica il suo aiuto al gruppo nella fase di sviluppo.
\end{itemize}


\subsection{C7: LLM: Assistente virtuale}
Il progetto prevede lo sviluppo di un Assistente Virtuale basato su modelli di Machine Learning per analizzare i dati aziendali e rispondere alle domande frequenti dei clienti riguardo ai prodotti disponibili, migliorando così l'interazione uomo-macchina e l'accessibilità delle informazioni. Utilizzando Large Language Models (LLM), il sistema si integrerà con un database relazionale e offrirà un'interfaccia utente mobile per facilitare la ricerca di informazioni in modo intuitivo e veloce.\\\\
\textbf{Pro}
\begin{itemize}
    \item Progetto sicuramente innovativo, rende più semplice il reperimento di informazioni da parte dell'utente.
    \item Capitolato esaustivo riguardo architettura e tecnologie da adottare.
    \item Molto supporto da parte dell'azienda.
\end{itemize}
\textbf{Contro}
\begin{itemize}
    \item Non sono specificati i requisiti di PoC e MVP.
\end{itemize}


\subsection{C8: Requirement Tracker - Plug-in VS Code}
Il progetto consiste nello sviluppo di un plug-in per Visual Studio Code, chiamato Requirement Tracker, che automatizza il tracciamento e la qualità dei requisiti di progetto, supportato da intelligenza artificiale per analisi e suggerimenti.\\\\
Il progetto presentato da questa azienda non ha riscontrato particolare interesse da parte del gruppo.


\subsection{C9: BuddyBot}
Il progetto mira a sviluppare un assistente virtuale, Buddy Bot, che sfrutta l'intelligenza artificiale per ottimizzare il knowledge management nelle aziende, facilitando l'accesso a informazioni critiche da diverse fonti tramite una chat in linguaggio naturale.\\\\
\textbf{Pro}
\begin{itemize}
    \item Progetto piuttosto utile riguardo all'accesso rapido alle informazioni provenienti da diversi fonti.
\end{itemize}
\textbf{Contro}
\begin{itemize}
    \item Requisiti di MVP non specificati chiaramente.
    \item L'azienda non fornisce pieno supporto durante tutte le fasi del progetto.
\end{itemize}
