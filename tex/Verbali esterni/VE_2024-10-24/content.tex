% Insert content here
\section{Informazioni generali}
\subsection{Dettagli sull'incontro}
\begin{itemize}
    \item \textbf{Luogo}: Modalità asicrona tramite condivisione del documento delle domande.
    \item \textbf{Data}: 24-10-2024
    \item \textbf{Ora di inizio}: -
    \item \textbf{Ora di fine}: -
\end{itemize}

\section{Motivo della riunione}
Richiesta di alcuni chiarimenti da parte del gruppo Six Bit Busters sul progetto \textbf{\textit{Vimar GENIALE}}.

\section{Resoconto delle domande}
Di seguito le domande richieste e una sintesi della risposta data dal proponente.\\
\begin{itemize}
    \item \textbf{Qual è l’obiettivo a lungo termine del sistema Vimar? È prevista un’integrazione con i vostri prodotti, oppure il progetto ha solamente lo scopo di dimostrare la potenzialità dei LLM applicato ai database?}\\\\
    L'azienda ha come obiettivo principale quello di dimostrare le potenzialità e i pregi dei LLM e, al contempo, comprenderne i limiti attraverso l'integrazione con i propri prodotti. Il progetto da noi sviluppato potrebbe essere proposto sia a personale interno che esterno all'azienda.\\
    
    \item \textbf{Nella presentazione non è stato dato particolare accento al PoC. Quali dovrebbero essere i principali criteri che esso deve rispettare?}\\\\
    Il PoC dovrà necessariamente implementare una componente per l'estrazione dei dati e per l'interrogazione sui prodotti Vimar, al fine di valutare la fattibilità tecnica del progetto. Tale implementazione includerà le API dell'applicativo server e una interfaccia grafica minima.\\
    
    \item \textbf{C'è un limite al numero di riunioni che è possibile richiedere per approfondire le tecnologie da voi richieste? }\\\\
    Il documento di capitolato non riporta limiti; Vimar è, anzi, disponibile a organizzare incontri per approfondire argomenti a noi sconosciuti o poco chiari, fornendo anche link a risorse attendibili e aggiornate a patto che venga richiesto in anticipo, si specifichino i dubbi e cosa ci aspettiamo di imparare. Inoltre, è previsto un incontro bisettimanale, fino al PoC, poi settimanale, dedicato a questioni specifiche e/o micro-approfondimenti.\\
\end{itemize}

\hfill\signature{Approvazione esterna}{Vimar}



