% Insert content here
\section{Informazioni generali}
\subsection{Dettagli sull`incontro}
\begin{itemize}
    \item \textbf{Luogo}: Call in Google Meet
    \item \textbf{Data}: 28-10-2024
    \item \textbf{Ora di inizio}: 15:00
    \item \textbf{Ora di fine}: 15:30
    \item \textbf{Partecipanti dell'azienda}: Stefano Dindo e Michele Massaro
    \item \textbf{Partecipanti del gruppo}: Tutti i componenti di \textit{Six Bit Busters} 
\end{itemize}

\section{Motivo della riunione}
Richiesta di chiarimenti tramite domande da parte del gruppo Six Bit Busters sul progetto proposto dall'azienda Var Group S.p.A.

\section{Resoconto delle domande}
Di seguito le domande richieste e una sintesi della risposta data dal proponente.

\subsection{Domande poste}
Il gruppo ha posto alcune domande inerenti al progetto per ottenere delucidazioni relative alla presentazione e al capitolato, al fine di accrescere l'interesse per il progetto e stabilire se potrebbe essere la nostra scelta per la candidatura.\\
\\
In particolare, sono state poste domande sul supporto che l’azienda offrirà prima e durante l'avanzamento del progetto e domande in merito alle tecnologie più complesse che ci richiedono di utilizzare (AWS e AI Agent) e alle modalità di comunicazione.\\
\\
Un altro punto discusso ha riguardato i software con cui il prodotto dovrà interagire e il grado di complessità che sarà necessario implementare per soddisfare le richieste dell'utente in linguaggio naturale.\\
\\
È stato poi chiesto quali funzionalità sono richieste al momento della realizzazione del PoC, e se potessimo ottenere maggiori dettagli rispetto a come l'interfaccia drag-and-drop dovrebbe apparire e funzionare.\\
\\
Come ultimo punto abbiamo chiesto se il progetto avrà una licenza privata oppure resterà un nostro progetto, e se è prevista una sua implementazione futura in azienda.

\subsection{Risposte dell'azienda}
L'azienda offre due sessioni di formazione in presenza per la parte di design thinking e per entrare nel dettaglio dei requisiti, porre dei limiti ed analizzarli, insegnandoci a svolgere un'attività di questo tipo.
Inoltre si è resa disponibile ad organizzare sessioni di formazione online qualora riscontrassimo difficoltà nell’utilizzo degli strumenti e delle tecnologie consigliati.\\
Per quanto riguarda la comunicazione asincrona, verrà messo a disposizione un canale Slack per porre domande e fare richieste.\\
Nel complesso il proponente risulta disponibile nei tempi e nelle modalità comunicative, aspetto fondamentale considerato dal gruppo per la scelta del capitolato.\\
\\
I software con cui il prodotto dovrà interagire verranno definiti nella sessione di design thinking (considerata dal gruppo fase molto importante messa a disposizione dall'azienda); in generale, essi si baseranno sull'utilizzo di software comuni come Word, Outlook e Calendar, oltre ad una piattaforma di messaggistica come Teams.
Lo scopo per il ostro gruppo sarà quello di riuscire a mappare l'idea di realizzazione e la fattibilità di questa soluzione riportando le difficoltà incontrate per realizzarlo.\\
\\
Ci è stato spiegato che il cuore del programma consiste nell'interpretazione delle richieste che l'utente farà in linguaggio naturale, sintetizzandole in operazioni automatizzate.
La parte complessa del progetto sarà la traduzione che avverrà dal linguaggio naturale ai blocchi e alle operazioni, che dovranno interagire tra loro, creando un workflow automatizzato.\\
\\
Per quanto riguarda il PoC ci è stato spiegato che le richieste non saranno troppo lontane da quelle del MVP. Aspetto considerato non particolarmente vantaggioso dal gruppo perchè lascia poco margine tra i due elementi.
In particolare, è necessario riuscire ad implementare le funzionalità minime limitandoci ad un singolo blocco (ossia un solo software, ad esempio Word) per dimostrare che la traduzione dal linguaggio naturale ad operazioni è fattibile e che è altresì possibile sfruttare le API per poi integrare i vari blocchi tra loro componendo dunque il prodotto finito.\\
\\
In merito al design dell'interfaccia drag-and-drop, ci sono stati suggeriti alcuni software già esistenti da cui trarre spunto.
Riguardo alla licenza e all'utilizzo finale del software ci è stato riferito che la licenza sarà privata ed il progetto verrà implementato nel software aziendale solo se la qualità del nostro prodotto sarà elevata.\\\\

\hfill\signature{Approvazione esterna}{Vargroup}