% Insert content here
\section{Informazioni generali}
\subsection{Dettagli sull'incontro}
\begin{itemize}
    \item \textbf{Luogo}: Call in Google Meet
    \item \textbf{Data}: 30-10-2024
    \item \textbf{Ora di inizio}: 17:00
    \item \textbf{Ora di fine}: 17:30
    \item \textbf{Partecipante dell'azienda}: Beggiato Alex
    \item \textbf{Partecipanti}:
    \begin{itemize}
        \item Bergamin Elia
        \item Diviesti Filippo
        \item Djossa Edgar
        \item Pincin Matteo 
        \item Soranzo Andrea  
    \end{itemize}
\end{itemize}

\section{Motivo della riunione}
Richiesta di chiarimenti tramite domande da parte del gruppo Six Bit Busters sul progetto proposto dall'azienda Sanmarco Informatica S.p.A.

\section{Resoconto delle domande}
Di seguito le domande richieste e una sintesi delle risposte date dal proponente.
\subsection{Domande poste}
Il gruppo ha posto alcune domande inerenti al progetto per ottenere delucidazioni relative alla presentazione e al capitolato, al fine di accrescere l'interesse per il progetto e stabilire se potrebbe essere la nostra scelta per la candidatura.\\
\\
Sono principalmente state poste domande sulle funzionalità essenziali che l'MVP deve presentare. In particolare sulla struttura generale della web app e sui dataset che essa deve utilizzare.\\
\\
In seguito sono state richieste dal gruppo ulteriori informazioni riguardanti il PoC e l'analisi dei requisiti.\\
\\
Successivamente sono stati richiesti chiarimenti sulla comunicazione sincrona e asincrona con particolare attenzione alle modalità con le quali queste si possono svolgere e con quale frequenza.\\
\\
Infine, l'ultimo punto di discussione ha riguardato eventuali funzionalità extra che il gruppo può implementare nel caso in cui venga soddisfatto l'MVP prima del termine finale del progetto.
\\
\subsection{Risposte dell'azienda}
Il prodotto finale deve prevedere una web app (escludendo concetti di responsive e accessibilità essendo l'ottica principale quella desktop) con landing page per la selezione del dataset reperito tramite richieste API predefinite scelte da un menù ristretto di fornitori. L'idea è quella di creare un prodotto che sia poi in futuro facilmente estendibile. La seconda pagina deve contenere una sezione per il grafico 3D navigabile, la tabella contenente il dataset e un menù con le principali funzionalità.
Inoltre è stato specificato che non si richiede alcuna funzione di login e registrazione utente.
Questo aspetto minimale è stato considerato dal gruppo un punto di forza del capitolato in quanto permette di concentrarsi pù sull'aspetto funzionale che su quello grafico e user-experience.\\
\\
In fase di PoC la web app deve funzionare sulla base di tre dataset definiti dal gruppo con tre dimensioni diverse con il target principale quello delle performance. Le tre dimensioni saranno definite nel documento di analisi dei requisiti nella sezione requisiti di performance. In particolare si intende concentrare l'attenzione sul tempo che l'utente è disposto ad aspettare per il caricamento del dataset. \\
\\
L'azienda offre la possibilità di stabilire all'inizio del progetto un intervallo di tempo che identifica la frequenza con la quale si possono organizzare degli incontri telematici con il gruppo. Inoltre la posta elettronica rimane la principale fonte di comunicazione asincrona. Queste riunioni periodiche son oggetto di interesse per il gruppo e pertanto molto apprezzate e importanti. \\
\\
In conclusione l'azienda ha suggerito alcune caratteristiche opzionali che il prodotto può presentare una volta completato l'MVP. In particolare consigliano di focalizzare l'attenzione sulle caratteristiche estetiche della web app come animazioni e colorazioni dinamiche del grafico in base al tipo di dato rappresentato.

\hfill\signature{Approvazione esterna}{Sanmarco Informatica}
