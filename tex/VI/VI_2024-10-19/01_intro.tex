% Insert content here
\section{Informazioni generali}
\subsection{Dettagli sull'incontro}
\begin{itemize}
    \item \textbf{Luogo}: Incontro in remoto su Discord
    \item \textbf{Data}: 19-10-2024
    \item \textbf{Ora di inizio}: 10:00
    \item \textbf{Ora di fine}: 11:30
    \item \textbf{Partecipanti}:
    \begin{itemize}
        \item Bergamin Elia
        \item Chilese Elena
        \item Diviesti Filippo
        \item Djossa Edgar
        \item Pincin Matteo 
        \item Soranzo Andrea  
    \end{itemize}
\end{itemize}

\section{Resoconto}
\subsection{Introduzione}
    Dopo aver chiarito alcune questioni organizzative, abbiamo approfonditamente analizzato i punti di forza e di debolezza di ciascun capitolato d'appalto. \\Infine abbiamo discusso dei possibili quesiti da porre ai proponenti dei capitolati di interesse del gruppo.
 
\subsection{Organizzazione incontri}
  Abbiamo deciso che, al fine di rimanere periodicamente aggiornati sul progetto, fosse necessario fissare almeno un incontro settimanale. Tali incontri sono stati programmati per il giovedì pomeriggio ed utilizzeremo la piattaforma \textit{Discord}, poiché le riunioni non hanno limiti di durata. Inoltre, grazie alla creazione di un canale apposito, possiamo avere diverse sezioni specifiche per la condivisione di link rilevanti, to-do list e tutto ciò che si renderà necessario durante lo sviluppo del progetto.
\subsection{Valutazioni dei Capitolati}
    Per maggiori dettagli si rimanda al documento specifico \textit{Studio di fattibilità} redatto prima di questo incontro ed approvato durante questo incontro.
\subsection{Quesiti per i proponenti}
    Ciascun membro ha espresso i propri dubbi da chiarire durante il colloquio con i referenti aziendali dei tre capitolati d'appalto che hanno suscitato il maggior interesse, ovvero C5, C3 e C2.
\section{Prossimi obiettivi}
    \begin{itemize}
        \item Organizzare un incontro con i proponenti
        \item Compilare la \textit{Dichiarazione degli impegni}
    \end{itemize}
